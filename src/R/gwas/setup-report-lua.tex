% standard document
%\begin{comment}
	\documentclass[a4paper,oneside,11pt]{article}
	\usepackage{amsmath,amssymb, amssymb, amsfonts, setspace, latexsym, epsfig, epsf, rotating, longtable, setspace, a4wide,verbatim, caption, hyperref}
	\usepackage{amsthm}
	\usepackage[utf8x]{inputenc}
	\usepackage{graphicx, grffile}

	\newcounter{global}
	%\setcounter{global}{1}
	\theoremstyle{definition} \newtheorem{exam}[global]{Example}%
	\theoremstyle{definition} \newtheorem{rem}[global]{Remark}%
	\theoremstyle{definition} \newtheorem{definition}[global]{Definition}%

	\theoremstyle{plain} \newtheorem{lemma}[global]{Lemma}%
	\theoremstyle{plain} \newtheorem{prop}[global]{Proposition}%
	\theoremstyle{plain} \newtheorem{them}[global]{Theorem}
	\theoremstyle{plain} \newtheorem{theorem}[global]{Theorem}%%
	\theoremstyle{plain} \newtheorem{cor}[global]{Corollary}%
%\end{comment}


%
%	<p> report template
%

\usepackage[tmargin=1in,bmargin=1in,lmargin=1.25in,rmargin=1.25in]{geometry}
\usepackage{fontspec}
\usepackage{xcolor}
\usepackage{titlesec}
\usepackage{datetime}
\usepackage{lastpage}
\defaultfontfeatures{Ligatures=TeX}
\setsansfont{Droid Sans}
%\setmainfont{CMU Serif}
\definecolor{Blue}{rgb}{.204,.353,.541}
\definecolor{LightBlue}{rgb}{.31,.506,.741}
\newfontfamily\subsubsectionfont[Color=LightBlue]{CMU Sans Serif}
\titleformat*{\section}{\Large\bfseries\sffamily\color{Blue}}
\titleformat*{\subsection}{\large\bfseries\sffamily\color{LightBlue}}
\titleformat*{\subsubsection}{\itshape\subsubsectionfont}
%\titleformat*{\subsubsection}{\itshape\sffamily\color{LightBlue}}

\newcommand{\analysisreport}[4]{
\begin{center}
	{\Large\bfseries\sffamily\color{Blue}
		Analysis report\\
		\vspace{1cm}
		\large #1\\
		\vspace{2cm}
	}
	\begin{tabular}{lp{8cm}}
	Analyst: 				& #2\\
	Project name:			& #1\\
	Consulting department:	& #3\\
	Contact email:			& #4\\
	Report generation:		& \today, \ \currenttime\\
	Number of pages: 		& \pageref{LastPage}
	\end{tabular}
\end{center}
\newpage
}

\begin{comment}
% allow spaces in file names
\makeatletter
\def\Ginclude@eps#1{%
  \message{}%
  \bgroup
  \def\@tempa{!}%
  \dimen@\Gin@req@width
  \dimen@ii.1bp\relax
  \divide\dimen@\dimen@ii
  \@tempdima\Gin@req@height
  \divide\@tempdima\dimen@ii
    \special{PSfile=#1\space
      llx=\Gin@llx\space
      lly=\Gin@lly\space
      urx=\Gin@urx\space
      ury=\Gin@ury\space
      \ifx\Gin@scalex\@tempa\else rwi=\number\dimen@\space\fi
      \ifx\Gin@scaley\@tempa\else rhi=\number\@tempdima\space\fi
      \ifGin@clip clip\fi}%
  \egroup
}
\makeatother
\end{comment}

% annals of applied statistics
\begin{comment}
	\documentclass[a4paper,oneside,11pt]{article}
	%\documentclass[referee]{biom}
	%\usepackage{biometrics2}
	\usepackage{amsmath,amssymb, amssymb, amsfonts, setspace, latexsym, epsfig, epsf, rotating, longtable, setspace, a4wide,verbatim, caption}
	% not needed for biometrics
	\usepackage{amsthm}
	\usepackage[utf8x]{inputenc}
	\usepackage{modnatbib}
	%\usepackage[latin1]{inputenc}

	\theoremstyle{plain} \newtheorem{cor}{Corollary}
	\theoremstyle{plain} \newtheorem{theorem}{Theorem}
	\theoremstyle{definition} \newtheorem{rem}{Remark}
	% biometrics
	\setlength{\voffset}{2cm}
	\setlength{\hoffset}{0cm}
	\setlength{\topmargin}{0cm}
	\setlength{\headheight}{0cm}
	\setlength{\headsep}{0cm}
	\setlength{\oddsidemargin}{0cm}
	\setlength{\evensidemargin}{0cm}
	% text height: 297 - 2 * 25.4 - 5
	\setlength{\textheight}{190mm}
	\setcounter{page}{1}
	\parindent = 0pt 
	%	natbib punctutation
	%\bibpunct{[}{]}{;}{a}{,}{;}
	%	citation commands
	%	\ct the standard cite (e.g. superscript number, of author name in parenthesis)
	%	\ctp a plain cite (e.g. superscript number, or author name w/o parenthesis)
	%	\ctt a text cite (e.g. author name even when superscript numbers are used)
	%	\ctf a standard cite with all authors (c.f. \ct)
	\newcommand{\ct}[1]{(\cite{#1})}
	\newcommand{\ctp}[1]{\cite{#1}}
	% %\newcommand{\ct}{\citep}
	%\newcommand{\ctt}{\citet}
	\newcommand{\ctt}[1]{\citet{#1}}
	\newcommand{\ctpr}[2]{(#1 \citet{#2})}
\end{comment}
% end annals of applied statistics

% biostatistics
\begin{comment}
	\documentclass{bio}
	\usepackage{dcolumn,color}
	\usepackage{amsmath,modamssymb,times,modlastpage,modamsfonts,modamsthm,modbm}
	\usepackage{graphicx,lscape,rotating,longtable,verbatim,caption}
	\usepackage[utf8x]{inputenc}
	\usepackage{modnatbib}
	\numberwithin{equation}{section}
	\pyear{2010}
	\jno{}

	\theoremstyle{definition} \newtheorem{exam}{Example}%
	%\theoremstyle{definition} \newtheorem{rem}{Remark}%
	\theoremstyle{definition} \newtheorem{definition}{Definition}%

	%\theoremstyle{plain} \newtheorem{lemma}{Lemma}%
	\theoremstyle{plain} \newtheorem{prop}{Proposition}%
	%\theoremstyle{plain} \newtheorem{them}{Theorem}
	%\theoremstyle{plain} \newtheorem{theorem}{Theorem}%%
	\theoremstyle{plain} \newtheorem{cor}{Corollary}%
	\theoremstyle{definition} \newtheorem{rem}{Remark}%

	% standard cite
	\newcommand{\ct}[1]{\citep{#1}}
	% cite in parantheses
	\newcommand{\ctp}[1]{\citet{#1}}
	% cite in the text
	\newcommand{\ctt}[1]{\citet{#1}}
	% cite with prefix (in paretheses
	\newcommand{\ctpr}[2]{(#1 \citet{#2})}
	\newcommand{\bibnamefont}[1]{\scshape #1 \normalfont}
\end{comment}
% end biostatistics

% biometrics
\begin{comment}
	\documentclass[a4paper,oneside,11pt]{article}
	%\documentclass[referee]{biom}
	%\usepackage{biometrics2}
	\usepackage{amsmath,amssymb, amssymb, amsfonts, setspace, latexsym, epsfig, epsf, rotating, longtable, setspace, a4wide,verbatim, caption}
	% not needed for biometrics
	\usepackage{amsthm}
	\usepackage[utf8x]{inputenc}
	%\usepackage{jasa}
	%\usepackage[latin1]{inputenc}

	% biometrics
	\setlength{\voffset}{2cm}
	\setlength{\hoffset}{0cm}
	\setlength{\topmargin}{0cm}
	\setlength{\headheight}{0cm}
	\setlength{\headsep}{0cm}
	\setlength{\oddsidemargin}{0cm}
	\setlength{\evensidemargin}{0cm}
	% text height: 297 - 2 * 25.4 - 5
	\setlength{\textheight}{190mm}
	\setcounter{page}{1}
	\parindent = 0pt 
	%	natbib punctutation
	\bibpunct{[}{]}{;}{a}{,}{;}
	%	citation commands
	%	\ct the standard cite (e.g. superscript number, of author name in parenthesis)
	%	\ctp a plain cite (e.g. superscript number, or author name w/o parenthesis)
	%	\ctt a text cite (e.g. author name even when superscript numbers are used)
	%	\ctf a standard cite with all authors (c.f. \ct)
	\newcommand{\ct}[1]{(\cite{#1})}
	\newcommand{\ctp}[1]{\cite{#1}}
	% %\newcommand{\ct}{\citep}
	\newcommand{\ctt}{\citet}
\end{comment}
% end biometrics


%jasa
\begin{comment}	% not needed for biometrics packages
	\documentclass[a4paper,oneside,11pt]{article}
	\usepackage{amsmath,amssymb, amssymb, amsfonts, setspace, latexsym, epsfig, epsf, rotating, longtable, setspace, a4wide,verbatim, caption}
	\usepackage{amsthm}
	\usepackage[utf8x]{inputenc}
	\usepackage{jasa}
	\theoremstyle{definition} \newtheorem{exam}{Example}%
	\theoremstyle{definition} \newtheorem{rem}{Remark}%
	\theoremstyle{definition} \newtheorem{definition}{Definition}%

	\theoremstyle{plain} \newtheorem{lemma}{Lemma}%
	\theoremstyle{plain} \newtheorem{prop}{Proposition}%
	\theoremstyle{plain} \newtheorem{them}{Theorem}
	\theoremstyle{plain} \newtheorem{theorem}{Theorem}%%
	\theoremstyle{plain} \newtheorem{cor}{Corollary}%
	\theoremstyle{definition} \newtheorem{rem}{Remark}%
\end{comment}
% end jasa

\newcommand{\sN}{\mathcal{N}}

\begin{comment}
\newcommand{\G}{\ensuremath{\mathcal{G}}}%
\newcommand{\R}{\ensuremath{\mathbb{R}}}%
\newcommand{\N}{\ensuremath{\mathbb{N}}}%
\newcommand{\FJ}{\ensuremath{\mathcal{J}_0^{-1}}}%
\newcommand{\FJI}{\ensuremath{\mathcal{J}_0}}%
\newcommand{\FI}{\ensuremath{\mathcal{I}_0^{-1}}}%
\newcommand{\FIjoi}{\ensuremath{\mathcal{I}_{Joint}^{-1}}}%
\newcommand{\FIjoor}{\ensuremath{\mathcal{I}_{Joint}}}%
\newcommand{\FIcom}{\ensuremath{\mathcal{I}_{comp}^{-1}}}%


\newcommand{\tr}{\mbox{tr}}
\newcommand{\vms}[1]{\mbox{\hspace{#1}}}
\newcommand{\Lower}[1]{\smash{\lower 1.5ex \hbox{#1}}}
\def\mapright#1{\mathbin{\overset{#1}{\longrightarrow}}}
\def\mapequal#1{\mathbin{\overset{#1}{=}}}
\def\pr{\mbox{pr}}
\def\lg{\mbox{logit }}   
\def\am{\mbox{ arg min } }
\def\refhg{\hangindent=20pt\hangafter=1}
\def\refmark{\par\vskip 2mm\noindent\refhg}
 
\newcommand{\indep}{\;\, \rule[0em]{.03em}{.6em} \hspace{-.25em}
\rule[0em]{.65em}{.03em} \hspace{-.25em} \rule[0em]{.03em}{.6em}\;\,}

\newcommand{\E}{\mathrm{E}}
\newcommand{\Es}{\mbox{\scriptsize{E}}}
\newcommand{\var}{\mbox{var}} %% Use for variances
\newcommand{\vecc}{\mbox{vec}}
\newcommand{\cov}{\mbox{cov}}
\newcommand{\rank}{\mbox{rank}}
\newcommand{\ols}{\mbox{{\sc los}}}
\newcommand{\sir}{\mbox{SIR}}
\newcommand{\cdf}{\mbox{{\sc cf}}}
\newcommand{\doc}{\mbox{{\sc doc}}}
\newcommand{\phd}{\mbox{p{\sc h}d}}
\newcommand{\e}{\mbox{$\varepsilon$}}
\newcommand{\greg}{\mbox{{\sc Greg}}}
\newcommand{\CERES}{\mbox{{\sc ceres}}}
\newcommand{\SAVE}{\mbox{{\sc save}}}
\newcommand{\save}{\mbox{{\sc save}}}
\newcommand{\sate}{\mbox{{\sc sate}}}
\newcommand{\diag}{\mbox{diag}}
\newcommand{\spc}{{\mathcal S}}

\newcommand{\0}{{\bf 0}}

\def\pr{\mbox{P}}
\newcommand{\spn}{\mbox{span}}
\newcommand{\scov}[1]{{\mathcal S}^{(#1)}_{{\mathrm Cove}}}
\newcommand{\MMM}{{\mathcal M}}
\newcommand{\X}{{X}} %% Use for BNF X
\newcommand{\x}{{x}} %% Use for BNF x
\newcommand{\Y}{{Y}}
\newcommand{\y}{y}
\newcommand{\el}{L}
\newcommand{\Z}{{Z}}
\newcommand{\grkp}{{\pi}}
\newcommand{\Q}{{Q}}
\newcommand{\z}{{z}}
\newcommand{\bb}{{b}}
\newcommand{\ba}{{a}}
\newcommand{\B}{{B}}
\newcommand{\A}{{A}}
\newcommand{\m}{{m}}
\newcommand{\br}{{r}}
\newcommand{\M}{{M}}
\newcommand{\K}{{K}}
\newcommand{\W}{{W}}
\newcommand{\F}{{F}}
\newcommand{\bk}{{k}}
\newcommand{\f}{{f}}
\newcommand{\D}{{\mbox{D}}}
\newcommand{\I}{{I}}
\newcommand{\U}{{\mbox{U}}}
\newcommand{\bu}{{u}}
\newcommand{\V}{{V}}
\newcommand{\ebs}{{e}}
\newcommand{\J}{{J}}
\newcommand{\bv}{{v}}
\newcommand{\Proj}{{P}}
\newcommand{\im}{i}
\newcommand{\p}{{P}}
\newcommand{\Rbf}{{\mbox{R}}}
\newcommand{\spcsave}{\spc^{\mathrm{\scriptscriptstyle SAVE}}}
\newcommand{\Spp}{\Sigma_{i}^{1/2}}
\newcommand{\Sm}{\Sigma_{i}^{-1/2}}
\newcommand{\simu}{\mathrm ism}
\newcommand{\Kbf}{K}
\newcommand{\Hbf}{{H}}

\newcommand{\vn}[1]{\mbox{\sl #1}}

\newcommand{\greekbold}[1]{\mbox{\boldmath $#1$}}
\newcommand{\betabf}{{\beta}}
\newcommand{\nubf}{{\nu}}
\newcommand{\etabf}{{\eta}}
\newcommand{\lambdabf}{{\lambda}}
\newcommand{\mubf}{{\mu}}
\newcommand{\deltabf}{{\delta}}
\newcommand{\Deltabf}{{\Delta}}
 
\newcommand{\alphabf}{{\alpha}}
\newcommand{\xibf}{{\xi}}
\newcommand{\gammabf}{{\gamma}}
\newcommand{\varepsilonbf}{{\varepsilon}}
\newcommand{\betahatbf}{\hat{\beta}}
\newcommand{\betahat}{\hat{\beta}}
\newcommand{\etahat} {\hat{\eta}}
\newcommand{\set}[1]{\{\,#1\,\}}
\end{comment}
